\documentclass[12pt]{article}
\usepackage[utf8]{inputenc}
\usepackage{amsmath, amssymb, amsfonts}
\usepackage{geometry}
\usepackage{graphicx}
\usepackage{fancyhdr}
\usepackage{titlesec}
\usepackage{setspace}
\usepackage{multicol}
\geometry{margin=2.5cm}
\usepackage{minted}

\begin{document}

    \begin{titlepage}
        \begin{center}
            \vspace*{2cm}

            \textbf{\Large UNIVERSIDADE POLITÉCNICA DE PERNAMBUCO}\\[0.6cm]
            \textbf{\large CURSO DE ENGENHARIA DE CONTROLE E AUTOMAÇÃO}\\[2cm]

            \textbf{\Huge TRABALHO DE CÁLCULO NUMÉRICO}\\[2cm]

            \textbf{\large Resolução de Sistemas de Equações Lineares por Eliminação de Gauss, Fatoração LU e Cholesky}\\[2cm]

            \begin{flushleft}
            \textbf{Professor:} Jornandes Dias da Silva\\[0.2cm]
            \textbf{Discentes:}\\
            Caio Cesar Leite Lima\\
            Gabriel Nobrega Toscano\\
            Ricardo Timoteo Wanderley
            \end{flushleft}

            \vfill

            \textbf{Recife, PE}\\
            \textbf{15 de Maio de 2025}

        \end{center}
    \end{titlepage}

    \tableofcontents
    \newpage


    \section{Sistema Original}
        Resolver o sistema de equações utilizando a técnica de Elininação de Gauss, com 4 casas decimais nos resultados:

        \[
        \begin{cases}
            2x_1 + 2x_2 + x_3 + x_4 = 7 \\
            x_1 - x_2 + 2x_3 - x_4 = 1 \\
            3x_1 + 2x_2 - 3x_3 - 2x_4 = 4\\
            4x_1 + 3x_2 + 2x_3 + x_4 = 12
        \end{cases};
        S_x^{(0)} = (x_1^{(0)} = 0.5000\ x_2^{(0)}= 2.0000,\ x_3^{(0)}= 0.9000\ x_4^{(0)}= 1.2000)^t;
        \]


    \section{Transformação do Sistema em Matriz}

        \[
        A = \begin{bmatrix}
            2 & 2 & 1 & 1 \\
            1 & -1 & 2 & -1 \\
            3 & 2 & -3 & - 2\\
            4 & 3 & 2 & 1
        \end{bmatrix}, \quad
        b = \begin{bmatrix}
            7 \\ 1 \\ 4 \\ 12
        \end{bmatrix};
        \]

    \section{Eliminação de Gauss}
        \subsection{Transformar em matriz triangular superior:}

            \texttt{Consiste em isolar as incógnitas para descobrir seus respectivos valores.}

            \subsubsection{Calcular multiplicadores (linha 1)}

                \texttt{Calcula o valor do multiplicador que é usado para zerar os elementos abaixo do pivô de cada linha.}

                \[
                m_{21} = \frac{1}{2} = 0.5000, \quad m_{31} = \frac{3}{2} = 1.5000, \quad m_{41} = \frac{4}{2} = 2.0000;
                \]

            \subsubsection{Zerar elementos abaixo do pivô (coluna 1)}

                \[
                \begin{bmatrix}
                    2 & 2 & 1 & 1 \\
                    1 & -1 & 2 & -1 \\
                    3 & 2 & -3 & - 2\\
                    4 & 3 & 2 & 1
                \end{bmatrix}
                \begin{bmatrix}
                    7 \\ 1 \\ 4 \\ 12
                \end{bmatrix};
                \begin{aligned}
                    L_2 &\leftarrow L_2 - m_{21} \cdot L_1 \\
                    L_3 &\leftarrow L_3 - m_{31} \cdot L_1
                \end{aligned};
                \]\\

                \[
                A^{(1)} = \begin{bmatrix}
                    2.0000 & 2.0000 & 1.0000 & 1.0000 \\
                    0 & -2.0000 & 1.5000 & -1.5000 \\
                    0 & -1.0000 & -4.5000 & -3.5000 \\
                    0 & -1.0000 & 0 & -1.0000
                \end{bmatrix},
                b^{(1)} = \begin{bmatrix}
                        7.0000 \\ -2.5000 \\ -6.5000 \\ -2.0000
                \end{bmatrix};
                \]

            \subsubsection{Calcular multiplicadores (linha 2)}

                \[
                m_{32} = \frac{-1}{-2} = 0.5000, \quad m_{42} = \frac{-1}{-2} = 0.5000;
                \]

            \subsubsection{Zerar elementos abaixo do pivô (coluna 2)}

                \[
                \begin{bmatrix}
                    2.0000 & 2.0000 & 1.0000 & 1.0000 \\
                    0 & -2.0000 & 1.5000 & -1.5000 \\
                    0 & -1.0000 & -4.5000 & -3.5000 \\
                    0 & -1.0000 & 0 & -1.0000
                \end{bmatrix}
                \begin{bmatrix}
                        7.0000 \\ -2.5000 \\ -6.5000 \\ -2.0000
                \end{bmatrix};
                \begin{aligned}
                    L_3 &\leftarrow L_3 - m_{32} \cdot L_2 \\
                    L_4 &\leftarrow L_4 - m_{42} \cdot L_2
                \end{aligned};
                \]\\

                \[
                A^{(2)} = \begin{bmatrix}
                    2.0000 & 2.0000 & 1.0000 & 1.0000 \\
                    0 & -2.0000 & 1.5000 & -1.5000 \\
                    0 & 0 & -5.2500 & -2.7500 \\
                    0 & 0 & -0.7500 & 0.2500
                \end{bmatrix},
                b^{(2)} = \begin{bmatrix}
                        7.0000 \\ -2.5000 \\ -5.2500 \\ -0.7500
                \end{bmatrix};
                \]\\

                \subsubsection{Calcular multiplicadores (linha 3)}

                \[
                m_{43} = \frac{-0.7500}{-5.2500} = 0.1428;
                \]

            \subsubsection{Zerar elementos abaixo do pivô (coluna 3)}

                \[
                \begin{bmatrix}
                    2.0000 & 2.0000 & 1.0000 & 1.0000 \\
                    0 & -2.0000 & 1.5000 & -1.5000 \\
                    0 & -1.0000 & -4.5000 & -3.5000 \\
                    0 & -1.0000 & 0 & -1.0000
                \end{bmatrix}
                \begin{bmatrix}
                        7.0000 \\ -2.5000 \\ -6.5000 \\ -2.0000
                \end{bmatrix};
                \begin{aligned}
                    L_4 &\leftarrow L_4 - m_{43} \cdot L_3
                \end{aligned};
                \]\\

                \[
                A^{(3)} = \begin{bmatrix}
                    2.0000 & 2.0000 & 1.0000 & 1.0000 \\
                    0 & -2.0000 & 1.5000 & -1.5000 \\
                    0 & 0 & -5.2500 & -2.7500 \\
                    0 & 0 & 0 & 0.6428
                \end{bmatrix},
                b^{(3)} = \begin{bmatrix}
                        7.0000 \\ -2.5000 \\ -5.2500 \\ 0
                \end{bmatrix};
                \]
                \newpage

        \subsection{Fazer as interação}

            \texttt{Consiste em isolar as incógnitas, fazendo as interações para descobrindo os valores de $x_1^{(k+1)}$, $x_2^{(k+1)}$, $x_3^{(k+1)}$ e $x_4^{(k+1)}$ }

            \subsubsection{Isolar as incógnitas}

                \begin{align*}
                    x_1^{(k+1)} &= \frac{1}{2} \cdot [7.0000 - 2.0000x_2^{(k)} - x_3^{(k)} - x_4^{(k)}]\\
                    x_2^{(k+1)} &= -\frac{1}{2} \cdot [-2.5000 - 1.5000x_3^{(k)} + 1.5000x_4^{(k)}]\\
                    x_3^{(k+1)} &= -\frac{1}{5.2500} \cdot [-5.2500 + 2.7500x_4^{(k)}]\\
                    x_4^{(k+1)} &= \frac{1}{0.6428} \cdot [0]
                \end{align*}

            \subsubsection{Primeira interação (k = 0)}

                \[
                S_x^{(0)} = (x_1^{(0)} = 0.5000\ x_2^{(0)}= 2.0000,\ x_3^{(0)}= 0.9000\ x_4^{(0)}= 1.2000)^t;
                \]

                \begin{align*}
                    x_1^{(1)} &= \frac{1}{2} \cdot [7.0000 - 2.0000x_2^{(0)} - x_3^{(0)} - x_4^{(0)}]\\
                    x_2^{(1)} &= -\frac{1}{2} \cdot [-2.5000 - 1.5000x_3^{(0)} + 1.5000x_4^{(0)}]\\
                    x_3^{(1)} &= -\frac{1}{5.2500} \cdot [-5.2500 + 2.7500x_4^{(0)}]\\
                    x_4^{(1)} &= \frac{1}{0.6428} \cdot [0]
                \end{align*}

                \begin{align*}
                    x_1^{(1)} = 0.7000\\ x_2^{(1)}= 0.6200\\ x_3^{(1)}= 0.3714\\ x_4^{(1)}= 0.0000
                \end{align*}
            \newpage

            \subsubsection{Segunda interação (k = 1)}

                \[
                S_x^{(1)} = (x_1^{(1)} = 0.7000\ x_2^{(1)}= 0.6200,\ x_3^{(1)}= 0.3714\ x_4^{(1)}= 0.0000)^t;
                \]

                \begin{align*}
                    x_1^{(2)} &= \frac{1}{2} \cdot [7.0000 - 2.0000x_2^{(1)} - x_3^{(1)} - x_4^{(1)}]\\
                    x_2^{(2)} &= -\frac{1}{2} \cdot [-2.5000 - 1.5000x_3^{(1)} + 1.5000x_4^{(1)}]\\
                    x_3^{(2)} &= -\frac{1}{5.2500} \cdot [-5.2500 + 2.7500x_4^{(1)}]\\
                    x_4^{(2)} &= \frac{1}{0.6428} \cdot [0]
                \end{align*}

                \begin{align*}
                    x_1^{(2)} = 2.6443\\ x_2^{(2)}= 1.5285\\ x_3^{(2)}= 1.0000\\ x_4^{(2)}= 0.0000
                \end{align*}

            \subsubsection{Terceira interação (k = 2)}

                \[
                S_x^{(2)} = (x_1^{(2)} = 2.6443\ x_2^{(2)}= 1.5285,\ x_3^{(2)}= 1.0000\ x_4^{(2)}= 0.0000)^t;
                \]

                \begin{align*}
                    x_1^{(3)} &= \frac{1}{2} \cdot [7.0000 - 2.0000x_2^{(2)} - x_3^{(2)} - x_4^{(2)}]\\
                    x_2^{(3)} &= -\frac{1}{2} \cdot [-2.5000 - 1.5000x_3^{(2)} + 1.5000x_4^{(2)}]\\
                    x_3^{(3)} &= -\frac{1}{5.2500} \cdot [-5.2500 + 2.7500x_4^{(2)}]\\
                    x_4^{(3)} &= \frac{1}{0.6428} \cdot [0]
                \end{align*}

                \begin{align*}
                    x_1^{(3)} = 1.4715\\ x_2^{(3)}= 2.0000\\ x_3^{(3)}= 1.0000\\ x_4^{(3)}= 0.0000
                \end{align*}

            \subsubsection{Quarta interação (k = 3)}

                \[
                S_x^{(3)} = (x_1^{(3)} = 1.4715\ x_2^{(3)}= 2.0000,\ x_3^{(3)}= 1.0000\ x_4^{(3)}= 0.0000)^t;
                \]

                \begin{align*}
                    x_1^{(4)} &= \frac{1}{2} \cdot [7.0000 - 2.0000x_2^{(3)} - x_3^{(3)} - x_4^{(3)}]\\
                    x_2^{(4)} &= -\frac{1}{2} \cdot [-2.5000 - 1.5000x_3^{(3)} + 1.5000x_4^{(3)}]\\
                    x_3^{(4)} &= -\frac{1}{5.2500} \cdot [-5.2500 + 2.7500x_4^{(3)}]\\
                    x_4^{(4)} &= \frac{1}{0.6428} \cdot [0]
                \end{align*}

                \begin{align*}
                    x_1^{(4)} = 1.0000\\ x_2^{(4)}= 2.0000\\ x_3^{(4)}= 1.0000\\ x_4^{(4)}= 0.0000
                \end{align*}

            \subsubsection{Quinta interação (k = 4)}

                \[
                S_x^{(4)} = (x_1^{(4)} = 1.0000\ x_2^{(4)}= 2.0000,\ x_3^{(4)}= 1.0000\ x_4^{(4)}= 0.0000)^t;
                \]

                \begin{align*}
                    x_1^{(5)} &= \frac{1}{2} \cdot [7.0000 - 2.0000x_2^{(4)} - x_3^{(4)} - x_4^{(4)}]\\
                    x_2^{(5)} &= -\frac{1}{2} \cdot [-2.5000 - 1.5000x_3^{(4)} + 1.5000x_4^{(4)}]\\
                    x_3^{(5)} &= -\frac{1}{5.2500} \cdot [-5.2500 + 2.7500x_4^{(4)}]\\
                    x_4^{(5)} &= \frac{1}{0.6428} \cdot [0]
                \end{align*}

                \begin{align*}
                    x_1^{(5)} = 1.0000\\ x_2^{(5)}= 2.0000\\ x_3^{(5)}= 1.0000\\ x_4^{(5)}= 0.0000
                \end{align*}


        \section{Soluções}

            \begin{align*}
                S_x^{(0)} = (x_1^{(0)} = 0.5000\ x_2^{(0)}= 2.0000,\ x_3^{(0)}= 0.9000\ x_4^{(0)}= 1.2000)^t;\\
                S_x^{(1)} = (x_1^{(1)} = 0.7000\ x_2^{(1)}= 0.6200,\ x_3^{(1)}= 0.3714\ x_4^{(1)}= 0.0000)^t;\\
                S_x^{(2)} = (x_1^{(2)} = 2.6443\ x_2^{(2)}= 1.5285,\ x_3^{(2)}= 1.0000\ x_4^{(2)}= 0.0000)^t;\\
                S_x^{(3)} = (x_1^{(3)} = 1.4715\ x_2^{(3)}= 2.0000,\ x_3^{(3)}= 1.0000\ x_4^{(3)}= 0.0000)^t;\\
                S_x^{(4)} = (x_1^{(4)} = 1.0000\ x_2^{(4)}= 2.0000,\ x_3^{(4)}= 1.0000\ x_4^{(4)}= 0.0000)^t;\\
                S_x^{(5)} = (x_1^{(5)} = 1.0000\ x_2^{(5)}= 2.0000,\ x_3^{(5)}= 1.0000\ x_4^{(5)}= 0.0000)^t;
            \end{align*}

        \section{Código de programação em Linguagem C}

            \begin{minted}[linenos, breaklines]{c}
#include <stdio.h>

#define N 4  /* Dimensao da matriz - NxN = 4x4 */

void gauss(double A[N][N], double b[N]);

int main()
{
     /* Matriz A (4x4) e vetor b */
    double A[N][N] = {
        {2, 2, 1, 1},
        {1, -1, 2, -1},
        {3, 2, -3, -2},
        {4, 3, 2, 1}
    };

    double b[N] = {7, 1, 4, 12};

     /* Resolver o sistema */
    gauss(A, b);

    return 0;
}

void gauss(double A[N][N], double b[N])
{
    int i, j, k;
    double multi;
    double x0[N];
    double x[N] = {0.5000, 2.0000, 0.9000, 1.2000};

     /* Eliminacao de Gauss (triangular superior) - printa na tela a matriz, seus multiplicadores e a operacao necessaria para tornar uma matriz triangular superior */
    for (i = 0; i < N - 1; i++)
    {
        for (j = i + 1; j < N; j++)
        {
            multi = A[j][i] / A[i][i];
            printf("M%d|%d = %.4f   L%d <- L%d - %.4fL%d \n", j + 1, i + 1, multi, j + 1, j + 1, multi, i + 1);
            for (k = 0; k < N; k++)
            {
                A[j][k] -= multi * A[i][k];
            }
            b[j] -= multi * b[i];
        }
        printf("\n");
        for (int linha = 0; linha < N; linha++)
        {
            printf("|");
            for (int coluna = 0; coluna < N; coluna++)
            {
                printf("%.4f ", A[linha][coluna]);
            }
            printf("| %.4f |\n", b[linha]);
        }
        printf("\n");
    }
    printf("\n\n");
    for (int s = 0; s < 6; s++)
    {
         /* Mostrar a solucao - Printa todas as 6 solucoes (Contando com a S-0) */
        printf("Solucao do sistema S-%d: ", s);
        for (i = 0; i < N; i++)
        {
            printf("X%d = %.4f ", i + 1, x[i]);
        }
        printf("\n\n");
         /* Substituicao regressiva - Descobre o valor das variaveis */
        x[N - 1] = b[N - 1] / A[N - 1][N - 1];
        for (i = N - 2; i >= 0; i--)
        {
            x0[i] = x[i];
            x[i] = b[i];
            for (j = i + 1; j < N; j++)
            {
                x[i] -= A[i][j] * x0[j];
            }
            x[i] /= A[i][i];
        }
    }
}
            \end{minted}

        \section{Retorno do código}


            \begin{verbatim}
M2|1 = 0.5000   L2 <- L2 - 0.5000L1
M3|1 = 1.5000   L3 <- L3 - 1.5000L1
M4|1 = 2.0000   L4 <- L4 - 2.0000L1

|2.0000 2.0000 1.0000 1.0000 | 7.0000 |
|0.0000 -2.0000 1.5000 -1.5000 | -2.5000 |
|0.0000 -1.0000 -4.5000 -3.5000 | -6.5000 |
|0.0000 -1.0000 0.0000 -1.0000 | -2.0000 |

M3|2 = 0.5000   L3 <- L3 - 0.5000L2
M4|2 = 0.5000   L4 <- L4 - 0.5000L2

|2.0000 2.0000 1.0000 1.0000 | 7.0000 |
|0.0000 -2.0000 1.5000 -1.5000 | -2.5000 |
|0.0000 0.0000 -5.2500 -2.7500 | -5.2500 |
|0.0000 0.0000 -0.7500 -0.2500 | -0.7500 |

M4|3 = 0.1429   L4 <- L4 - 0.1429L3

|2.0000 2.0000 1.0000 1.0000 | 7.0000 |
|0.0000 -2.0000 1.5000 -1.5000 | -2.5000 |
|0.0000 0.0000 -5.2500 -2.7500 | -5.2500 |
|0.0000 0.0000 0.0000 0.1429 | 0.0000 |



Solucao do sistema S-0: X1 = 0.5000 X2 = 2.0000 X3 = 0.9000 X4 = 1.2000

Solucao do sistema S-1: X1 = 1.0500 X2 = 1.9250 X3 = 1.0000 X4 = 0.0000

Solucao do sistema S-2: X1 = 1.0750 X2 = 2.0000 X3 = 1.0000 X4 = 0.0000

Solucao do sistema S-3: X1 = 1.0000 X2 = 2.0000 X3 = 1.0000 X4 = 0.0000

Solucao do sistema S-4: X1 = 1.0000 X2 = 2.0000 X3 = 1.0000 X4 = 0.0000

Solucao do sistema S-5: X1 = 1.0000 X2 = 2.0000 X3 = 1.0000 X4 = 0.0000

            \end{verbatim}

\end{document}


