\documentclass[12pt,a4paper]{article}

\usepackage[utf8]{inputenc}
\usepackage[T1]{fontenc}
\usepackage[brazil]{babel}

\usepackage{amsmath, amsfonts, amssymb}

\usepackage{times}
\usepackage{geometry}
\geometry{a4paper, left=3cm, top=3cm, right=2cm, bottom=2cm}

\usepackage{setspace}
\onehalfspacing
\usepackage{ragged2e}
\usepackage{parskip}
\usepackage{ulem}

\usepackage{graphicx}
\usepackage{fancyhdr}
\pagestyle{fancy}
\fancyhf{}
\fancyfoot[R]{\thepage}
\renewcommand{\headrulewidth}{0pt}
\usepackage{titlesec}
\usepackage{multicol}
\usepackage{minted}

\begin{document}

\begin{titlepage}
\begin{center}
    \large
    \textbf{UNIVERSIDADE DE PERNAMBUCO - UPE}\\
    \textbf{ESCOLA POLITÉCNICA DE PERNAMBUCO - POLI}\\
    \textbf{BACHARELADO EM ENGENHARIA DE CONTROLE E AUTOMAÇÃO}\\[3.5cm]

    ALUNOS:\\
    CAIO CÉSAR LEITE DE LIMA\\
    GABRIEL NÓBREGA TOSCANO\\
    RICARDO TIMOTEO WANDERLEY\\[2cm]

    \textbf{\Large CÁLCULO NUMÉRICO:\\ SISTEMA DE EQUAÇÕES ALGÉBRICAS LINEARES}\\
    \text{\Large RESOLUÇÃO DE SEALS POR MÉTODOS DIRETOS}
\end{center}

\vspace*{\fill}
\begin{center}
    RECIFE - PE\\
    MAIO DE 2025
\end{center}
\end{titlepage}

\begin{titlepage}
\begin{center}
    \large
    ALUNOS:\\
    CAIO CÉSAR LEITE DE LIMA\\
    GABRIEL NÓBREGA TOSCANO\\
    RICARDO TIMOTEO WANDERLEY\\[2cm]

    TURMA AT\\[2cm]

    \textbf{\Large CÁLCULO NUMÉRICO:\\ SISTEMA DE EQUAÇÕES ALGÉBRICAS LINEARES}\\
    \text{\Large RESOLUÇÃO DE SEALS POR MÉTODOS DIRETOS}
\end{center}

\vspace{1cm}

\begin{flushright}
\begin{minipage}{0.5\textwidth}
\justifying
Atividade voltada à resolução de Sistemas de Equações Algébricas Lineares (SEALs), utilizando métodos diretos: Gauss, Fatoração LU e Cholesky. Solicitação do Prof. Dr. Jornandes Dias, na disciplina de Cálculo Numérico, como parte da avaliação da 1ª unidade.
\end{minipage}
\end{flushright}

\vspace*{\fill}
\begin{center}
    RECIFE - PE\\
    MAIO DE 2025
\end{center}
\end{titlepage}

\tableofcontents
\newpage

\noindent \section{1ª QUESTÃO} (0,5 PONTO): Resolver o sistema de equações utilizando a técnica de Eliminação de Gauss, com 4 casas decimais nos resultados.

\[
\begin{cases}
2x_1 + 2x_2 + x_3 + x_4 = 7 \\
x_1 - x_2 + 2x_3 - x_4 = 1 \\
3x_1 + 2x_2 - 3x_3 - 2x_4 = 4 \\
4x_1 + 3x_2 + 2x_3 + x_4 = 12
\end{cases}
\]
\[
S_x^{(0)} = (x_1^{(0)} = 0.5000,\ x_2^{(0)} = 2.0000,\ x_3^{(0)} = 0.9000,\ x_4^{(0)} = 1.2000)^t;
\]

\textbf{A.1)} Mostrar a metodologia de eliminação de Gauss.

\textbf{A.2)} Fazer 6 soluções.

\vspace{0.5cm}

\subsection{Transformação do Sistema em Matriz}

\[
A = \begin{bmatrix}
2 & 2 & 1 & 1 \\
1 & -1 & 2 & -1 \\
3 & 2 & -3 & -2 \\
4 & 3 & 2 & 1
\end{bmatrix}, \quad
b = \begin{bmatrix}
7 \\ 1 \\ 4 \\ 12
\end{bmatrix};
\]

\subsection{Eliminação de Gauss}

\subsubsection{Transformar em matriz triangular superior:}

Consiste em isolar as incógnitas para descobrir seus respectivos valores.

\subsubsection{Calcular multiplicadores (linha 1)}

Calcula o valor do multiplicador que é usado para zerar os elementos abaixo do pivô de cada linha.

\[
m_{21} = \frac{1}{2} = 0.5000, \quad m_{31} = \frac{3}{2} = 1.5000, \quad m_{41} = \frac{4}{2} = 2.0000;
\]

\subsubsection{Zerar elementos abaixo do pivô (coluna 1)}

\[
\begin{aligned}
L_2 &\leftarrow L_2 - m_{21} \cdot L_1 \\
L_3 &\leftarrow L_3 - m_{31} \cdot L_1
\end{aligned}
\]

\[
A^{(1)} = \begin{bmatrix}
2.0000 & 2.0000 & 1.0000 & 1.0000 \\
0 & -2.0000 & 1.5000 & -1.5000 \\
0 & -1.0000 & -4.5000 & -3.5000 \\
0 & -1.0000 & 0 & -1.0000
\end{bmatrix},
\quad
b^{(1)} = \begin{bmatrix}
7.0000 \\ -2.5000 \\ -6.5000 \\ -2.0000
\end{bmatrix};
\]

\subsubsection{Calcular multiplicadores (linha 2)}

\[
m_{32} = \frac{-1}{-2} = 0.5000, \quad m_{42} = \frac{-1}{-2} = 0.5000;
\]

\subsubsection{Zerar elementos abaixo do pivô (coluna 2)}

\[
\begin{aligned}
L_3 &\leftarrow L_3 - m_{32} \cdot L_2 \\
L_4 &\leftarrow L_4 - m_{42} \cdot L_2
\end{aligned}
\]

\[
A^{(2)} = \begin{bmatrix}
2.0000 & 2.0000 & 1.0000 & 1.0000 \\
0 & -2.0000 & 1.5000 & -1.5000 \\
0 & 0 & -5.2500 & -2.7500 \\
0 & 0 & -0.7500 & 0.2500
\end{bmatrix},
\quad
b^{(2)} = \begin{bmatrix}
7.0000 \\ -2.5000 \\ -5.2500 \\ -0.7500
\end{bmatrix};
\]

\subsubsection{Calcular multiplicadores (linha 3)}

\[
m_{43} = \frac{-0.7500}{-5.2500} = 0.1428;
\]

\subsubsection{Zerar elementos abaixo do pivô (coluna 3)}

\[
L_4 \leftarrow L_4 - m_{43} \cdot L_3
\]

\[
A^{(3)} = \begin{bmatrix}
2.0000 & 2.0000 & 1.0000 & 1.0000 \\
0 & -2.0000 & 1.5000 & -1.5000 \\
0 & 0 & -5.2500 & -2.7500 \\
0 & 0 & 0 & 0.6428
\end{bmatrix},
\quad
b^{(3)} = \begin{bmatrix}
7.0000 \\ -2.5000 \\ -5.2500 \\ 0
\end{bmatrix};
\]

\subsection{Fazer as interações}

Consiste em isolar as incógnitas, fazendo as interações para descobrir os valores de $x_1^{(k+1)}$, $x_2^{(k+1)}$, $x_3^{(k+1)}$ e $x_4^{(k+1)}$.

\subsubsection{Isolar as incógnitas}


                \begin{align*}
                    x_1^{(k+1)} &= \frac{1}{2} \cdot [7.0000 - 2.0000x_2^{(k)} - x_3^{(k)} - x_4^{(k)}]\\
                    x_2^{(k+1)} &= -\frac{1}{2} \cdot [-2.5000 - 1.5000x_3^{(k)} + 1.5000x_4^{(k)}]\\
                    x_3^{(k+1)} &= -\frac{1}{5.2500} \cdot [-5.2500 + 2.7500x_4^{(k)}]\\
                    x_4^{(k+1)} &= \frac{1}{0.6428} \cdot [0]
                \end{align*}

            \subsubsection{Primeira interação (k = 0)}

                \[
                S_x^{(0)} = (x_1^{(0)} = 0.5000\ x_2^{(0)}= 2.0000,\ x_3^{(0)}= 0.9000\ x_4^{(0)}= 1.2000)^t;
                \]

                \begin{align*}
                    x_1^{(1)} &= \frac{1}{2} \cdot [7.0000 - 2.0000x_2^{(0)} - x_3^{(0)} - x_4^{(0)}]\\
                    x_2^{(1)} &= -\frac{1}{2} \cdot [-2.5000 - 1.5000x_3^{(0)} + 1.5000x_4^{(0)}]\\
                    x_3^{(1)} &= -\frac{1}{5.2500} \cdot [-5.2500 + 2.7500x_4^{(0)}]\\
                    x_4^{(1)} &= \frac{1}{0.6428} \cdot [0]
                \end{align*}
                \begin{align*}
                    x_1^{(1)} = 0.7000\\ x_2^{(1)}= 0.6200\\ x_3^{(1)}= 0.3714\\ x_4^{(1)}= 0.0000
                \end{align*}

            \subsubsection{Segunda interação (k = 1)}

                \[
                S_x^{(1)} = (x_1^{(1)} = 0.7000\ x_2^{(1)}= 0.6200,\ x_3^{(1)}= 0.3714\ x_4^{(1)}= 0.0000)^t;
                \]
                \begin{align*}
                    x_1^{(2)} &= \frac{1}{2} \cdot [7.0000 - 2.0000x_2^{(1)} - x_3^{(1)} - x_4^{(1)}]\\
                    x_2^{(2)} &= -\frac{1}{2} \cdot [-2.5000 - 1.5000x_3^{(1)} + 1.5000x_4^{(1)}]\\
                    x_3^{(2)} &= -\frac{1}{5.2500} \cdot [-5.2500 + 2.7500x_4^{(1)}]\\
                    x_4^{(2)} &= \frac{1}{0.6428} \cdot [0]
                \end{align*}
                \begin{align*}
                    x_1^{(2)} = 2.6443\\ x_2^{(2)}= 1.5285\\ x_3^{(2)}= 1.0000\\ x_4^{(2)}= 0.0000
                \end{align*}

            \subsubsection{Terceira interação (k = 2)}

                \[
                S_x^{(2)} = (x_1^{(2)} = 2.6443\ x_2^{(2)}= 1.5285,\ x_3^{(2)}= 1.0000\ x_4^{(2)}= 0.0000)^t;
                \]

                \begin{align*}
                    x_1^{(3)} &= \frac{1}{2} \cdot [7.0000 - 2.0000x_2^{(2)} - x_3^{(2)} - x_4^{(2)}]\\
                    x_2^{(3)} &= -\frac{1}{2} \cdot [-2.5000 - 1.5000x_3^{(2)} + 1.5000x_4^{(2)}]\\
                    x_3^{(3)} &= -\frac{1}{5.2500} \cdot [-5.2500 + 2.7500x_4^{(2)}]\\
                    x_4^{(3)} &= \frac{1}{0.6428} \cdot [0]
                \end{align*}
                \begin{align*}
                    x_1^{(3)} = 1.4715\\ x_2^{(3)}= 2.0000\\ x_3^{(3)}= 1.0000\\ x_4^{(3)}= 0.0000
                \end{align*}

            \subsubsection{Quarta interação (k = 3)}

                \[
                S_x^{(3)} = (x_1^{(3)} = 1.4715\ x_2^{(3)}= 2.0000,\ x_3^{(3)}= 1.0000\ x_4^{(3)}= 0.0000)^t;
                \]

                \begin{align*}
                    x_1^{(4)} &= \frac{1}{2} \cdot [7.0000 - 2.0000x_2^{(3)} - x_3^{(3)} - x_4^{(3)}]\\
                    x_2^{(4)} &= -\frac{1}{2} \cdot [-2.5000 - 1.5000x_3^{(3)} + 1.5000x_4^{(3)}]\\
                    x_3^{(4)} &= -\frac{1}{5.2500} \cdot [-5.2500 + 2.7500x_4^{(3)}]\\
                    x_4^{(4)} &= \frac{1}{0.6428} \cdot [0]
                \end{align*}
                \begin{align*}
                    x_1^{(4)} = 1.0000\\ x_2^{(4)}= 2.0000\\ x_3^{(4)}= 1.0000\\ x_4^{(4)}= 0.0000
                \end{align*}

            \subsubsection{Quinta interação (k = 4)}

                \[
                S_x^{(4)} = (x_1^{(4)} = 1.0000\ x_2^{(4)}= 2.0000,\ x_3^{(4)}= 1.0000\ x_4^{(4)}= 0.0000)^t;
                \]
                \begin{align*}
                    x_1^{(5)} &= \frac{1}{2} \cdot [7.0000 - 2.0000x_2^{(4)} - x_3^{(4)} - x_4^{(4)}]\\
                    x_2^{(5)} &= -\frac{1}{2} \cdot [-2.5000 - 1.5000x_3^{(4)} + 1.5000x_4^{(4)}]\\
                    x_3^{(5)} &= -\frac{1}{5.2500} \cdot [-5.2500 + 2.7500x_4^{(4)}]\\
                    x_4^{(5)} &= \frac{1}{0.6428} \cdot [0]
                \end{align*}
                \begin{align*}
                    x_1^{(5)} = 1.0000\\ x_2^{(5)}= 2.0000\\ x_3^{(5)}= 1.0000\\ x_4^{(5)}= 0.0000
                \end{align*}

        \subsection{Soluções}

            \begin{align*}
                S_x^{(0)} = (x_1^{(0)} = 0.5000\ x_2^{(0)}= 2.0000,\ x_3^{(0)}= 0.9000\ x_4^{(0)}= 1.2000)^t;\\
                S_x^{(1)} = (x_1^{(1)} = 0.7000\ x_2^{(1)}= 0.6200,\ x_3^{(1)}= 0.3714\ x_4^{(1)}= 0.0000)^t;\\
                S_x^{(2)} = (x_1^{(2)} = 2.6443\ x_2^{(2)}= 1.5285,\ x_3^{(2)}= 1.0000\ x_4^{(2)}= 0.0000)^t;\\
                S_x^{(3)} = (x_1^{(3)} = 1.4715\ x_2^{(3)}= 2.0000,\ x_3^{(3)}= 1.0000\ x_4^{(3)}= 0.0000)^t;\\
                S_x^{(4)} = (x_1^{(4)} = 1.0000\ x_2^{(4)}= 2.0000,\ x_3^{(4)}= 1.0000\ x_4^{(4)}= 0.0000)^t;\\
                S_x^{(5)} = (x_1^{(5)} = 1.0000\ x_2^{(5)}= 2.0000,\ x_3^{(5)}= 1.0000\ x_4^{(5)}= 0.0000)^t;
            \end{align*}

        \newpage

        \subsection{Código de programação em Linguagem C}

            \begin{minted}[linenos, breaklines]{c}
#include <stdio.h>

#define N 4  /* Dimensao da matriz - NxN = 4x4 */

void gauss(double A[N][N], double b[N]);

int main()
{
     /* Matriz A (4x4) e vetor b */
    double A[N][N] = {
        {2, 2, 1, 1},
        {1, -1, 2, -1},
        {3, 2, -3, -2},
        {4, 3, 2, 1}
    };

    double b[N] = {7, 1, 4, 12};

     /* Resolver o sistema */
    gauss(A, b);

    return 0;
}

void gauss(double A[N][N], double b[N])
{
    int i, j, k;
    double multi;
    double x0[N];
    double x[N] = {0.5000, 2.0000, 0.9000, 1.2000};

     /* Eliminacao de Gauss (triangular superior) - printa na tela a matriz, seus multiplicadores e a operacao necessaria para tornar uma matriz triangular superior */
    for (i = 0; i < N - 1; i++)
    {
        for (j = i + 1; j < N; j++)
        {
            multi = A[j][i] / A[i][i];
            printf("M%d|%d = %.4f   L%d <- L%d - %.4fL%d \n", j + 1, i + 1, multi, j + 1, j + 1, multi, i + 1);
            for (k = 0; k < N; k++)
            {
                A[j][k] -= multi * A[i][k];
            }
            b[j] -= multi * b[i];
        }
        printf("\n");
        for (int linha = 0; linha < N; linha++)
        {
            printf("|");
            for (int coluna = 0; coluna < N; coluna++)
            {
                printf("%.4f ", A[linha][coluna]);
            }
            printf("| %.4f |\n", b[linha]);
        }
        printf("\n");
    }
    printf("\n\n");
    for (int s = 0; s < 6; s++)
    {
         /* Mostrar a solucao - Printa todas as 6 solucoes (Contando com a S-0) */
        printf("Solucao do sistema S-%d: ", s);
        for (i = 0; i < N; i++)
        {
            printf("X%d = %.4f ", i + 1, x[i]);
        }
        printf("\n\n");
         /* Substituicao regressiva - Descobre o valor das variaveis */
        x[N - 1] = b[N - 1] / A[N - 1][N - 1];
        for (i = N - 2; i >= 0; i--)
        {
            x0[i] = x[i];
            x[i] = b[i];
            for (j = i + 1; j < N; j++)
            {
                x[i] -= A[i][j] * x0[j];
            }
            x[i] /= A[i][i];
        }
    }
}
            \end{minted}

\subsection{Retorno do código}

\begin{minted}[fontsize=\normalsize, breaklines]{text}
M2|1 = 0.5000   L2 <- L2 - 0.5000L1
M3|1 = 1.5000   L3 <- L3 - 1.5000L1
M4|1 = 2.0000   L4 <- L4 - 2.0000L1

|2.0000 2.0000 1.0000 1.0000 | 7.0000 |
|0.0000 -2.0000 1.5000 -1.5000 | -2.5000 |
|0.0000 -1.0000 -4.5000 -3.5000 | -6.5000 |
|0.0000 -1.0000 0.0000 -1.0000 | -2.0000 |

M3|2 = 0.5000   L3 <- L3 - 0.5000L2
M4|2 = 0.5000   L4 <- L4 - 0.5000L2

|2.0000 2.0000 1.0000 1.0000 | 7.0000 |
|0.0000 -2.0000 1.5000 -1.5000 | -2.5000 |
|0.0000 0.0000 -5.2500 -2.7500 | -5.2500 |
|0.0000 0.0000 -0.7500 -0.2500 | -0.7500 |

M4|3 = 0.1429   L4 <- L4 - 0.1429L3

|2.0000 2.0000 1.0000 1.0000 | 7.0000 |
|0.0000 -2.0000 1.5000 -1.5000 | -2.5000 |
|0.0000 0.0000 -5.2500 -2.7500 | -5.2500 |
|0.0000 0.0000 0.0000 0.1429 | 0.0000 |

Solucao do sistema S-0: X1 = 0.5000 X2 = 2.0000 X3 = 0.9000 X4 = 1.2000
Solucao do sistema S-1: X1 = 1.0500 X2 = 1.9250 X3 = 1.0000 X4 = 0.0000
Solucao do sistema S-2: X1 = 1.0750 X2 = 2.0000 X3 = 1.0000 X4 = 0.0000
Solucao do sistema S-3: X1 = 1.0000 X2 = 2.0000 X3 = 1.0000 X4 = 0.0000
Solucao do sistema S-4: X1 = 1.0000 X2 = 2.0000 X3 = 1.0000 X4 = 0.0000
Solucao do sistema S-5: X1 = 1.0000 X2 = 2.0000 X3 = 1.0000 X4 = 0.0000
\end{minted}

\newpage

\noindent \section{2ª QUESTÃO} (1,0 PONTOS): Resolver o sistema de equações utilizando a técnica de fatoração LU, admitindo 4 casas decimais nos resultados.

\[
\left\{
\begin{aligned}
2x_1^{(k)} + 2x_2^{(k)} + 6x_3^{(k)} &= -7 \\
2x_1^{(k)} + 2x_2^{(k)} - x_3^{(k)} &= 7 \\
6x_1^{(k)} - x_2^{(k)} + 12x_3^{(k)} &= -2
\end{aligned}
\right.
\]

\[
\boldsymbol{S_x^{(0)}} =
\left(
\boldsymbol{x_1^{(0)}} = 0.2000, \quad
\boldsymbol{x_2^{(0)}} = 1.0000, \quad
\boldsymbol{x_3^{(0)}} = 0.7000
\right)
\]

\[
\boldsymbol{S_y^{(0)}} =
\left(
\boldsymbol{y_1^{(0)}} = 0.0000, \quad
\boldsymbol{y_2^{(0)}} = 0.0000, \quad
\boldsymbol{y_3^{(0)}} = 0.0000
\right)
\]

\textbf{A.1)} Mostrar a metodologia de eliminação de Gauss.

\textbf{A.2)} Fazer 6 soluções.

\vspace{0.5cm}

(*) Transformar de Sistema para Matriz
\[
A =
\begin{bmatrix}
2 & 2 & 6 \\
2 & 2 & -1 \\
6 & -1 & 12
\end{bmatrix},
\quad
b =
\begin{bmatrix}
-7 \\
7 \\
-2
\end{bmatrix}
\]

\subsection{Pivoteamento Parcial}

Aplicamos permutação para que o maior elemento da coluna 1 (em módulo) fique na diagonal. Trocamos a linha 1 com a linha 3:

\[
PA =
\begin{bmatrix}
6 & -1 & 12 \\
2 & 2 & -1 \\
2 & 2 & 6
\end{bmatrix},
\quad
Pb =
\begin{bmatrix}
-2 \\
7 \\
-7
\end{bmatrix}
\]

\subsection{Eliminação de Gauss (Fatoração LU)}

\subsubsection{Zerando abaixo do pivô (coluna 1)}

Multiplicadores:

\[
m_{21} = \frac{a_{21}}{a_{11}},\quad
m_{31} = \frac{a_{31}}{a_{11}}
\]

\[
m_{21} = \frac{2}{6} = 0.3333,\quad
m_{31} = \frac{2}{6} = 0.3333
\]

Atualizando linhas:
\begin{align*}
L_2 &\leftarrow L_2 - m_{21} \cdot L_1 \\
L_3 &\leftarrow L_3 - m_{31} \cdot L_1
\end{align*}
\[
\Rightarrow
\text{$A^{(1)} =$}
\begin{bmatrix}
6.0000 & -1.0000 & 12.0000 \\
0      & 2.3333  & -5.0000 \\
0      & 2.3333  & 2.0000
\end{bmatrix}
\]

\subsubsection{Zerando elemento (3,2)}

\[
m_{32} = \frac{a_{32}}{a_{22}}
\]

\[
m_{32} = \frac{2.3333}{2.3333} = 1.0000
\]

\[
L_3 \leftarrow L_3 - m_{32} \cdot L_2
\]

\[
\Rightarrow
\text{$A^{(2)} =$}
\begin{bmatrix}
6.0000 & -1.0000 & 12.0000 \\
0      & 2.3333  & -5.0000 \\
0      & 0       & 7.0000
\end{bmatrix}
\]

\subsection{Matrizes L e U}

\[
L =
\begin{bmatrix}
1      & 0      & 0 \\
m_{21}^{(k+1)} & 1      & 0 \\
m_{31}^{(k+1)} & m_{32}^{(k+1)} & 1
\end{bmatrix},
\quad
U =
\begin{bmatrix}
a_{11}^{(k)} & a_{12}^{(k)} & a_{13}^{(k)} \\
0      & a_{22}^{(k)}  & a_{23}^{(k)} \\
0      & 0       & a_{33}^{(k)}
\end{bmatrix}
\]

\vspace{1cm}

\[
L =
\begin{bmatrix}
1      & 0      & 0 \\
0.3333 & 1      & 0 \\
0.3333 & 1.0000 & 1
\end{bmatrix},
\quad
U =
\begin{bmatrix}
6.0000 & -1.0000 & 12.0000 \\
0      & 2.3333  & -5.0000 \\
0      & 0       & 7.0000
\end{bmatrix}
\]

\subsection{Metodologia}

\begin{align*}
& L_{ij} \cdot y_j^{(k)} = b_j \\
& U_{ij} \cdot x_j^{(k)} = y_j^{(k)}
\end{align*}

\begin{equation*}
\left\{
\begin{aligned}
y_1^{(k+1)} &= b_1 \\
y_2^{(k+1)} &= b_2 - m_{21}^{(k+1)} y_1^{(k)} \\
y_3^{(k+1)} &= b_3 - m_{31}^{(k+1)} y_1^{(k)} - m_{32}^{(k+1)}y_2^{(k)}
\end{aligned}
\right.
\end{equation*}

\vspace{1cm}

\begin{equation*}
\left\{
\begin{aligned}
x_1^{(k+1)} &= \frac{1}{a_{11}}[y_1^{(k+1)} - ( a_{12}x_2^{(k)} + a_{13}x_3^{(k)})] \\
x_2^{(k+1)} &= \frac{1}{a_{22}}[y_2^{(k+1)} - a_{23}x_3^{(k)}]  \\
x_3^{(k+1)} &= \frac{1}{a_{33}}[y_3^{(k+1)}]
\end{aligned}
\right.
\end{equation*}

\subsection{Resolvendo $Ly = Pb$}

\[
Ly =
\begin{bmatrix}
1      & 0      & 0 \\
0.3333 & 1      & 0 \\
0.3333 & 1.0000 & 1
\end{bmatrix}
\begin{bmatrix}
y_1^{(1)} \\
y_2^{(1)} \\
y_3^{(1)}
\end{bmatrix}
=
\begin{bmatrix}
-2 \\
7 \\
-7
\end{bmatrix}
\]

\begin{equation*}
\left\{
\begin{aligned}
y_1^{(k+1)} &= -2 \\
y_2^{(k+1)} &= 7 - 0.3333 \cdot y_1^{(k)} \\
y_3^{(k+1)} &= -7 - 0.3333 \cdot y_1^{(k)} - y_2^{(k)}
\end{aligned}
\right.
\end{equation*}

\vspace{0,3 cm}

Isolando $y^{(1)}$, para k = 0, obtemos:
\begin{align*}
\boldsymbol{y_1^{(1)}} &= -2 \\
0.3333 \cdot (0) + y_2^{(1)} &= 7 \Rightarrow \boldsymbol{y_2^{(1)}} = 7.0000 \\
0.3333 \cdot (0) + 1.0000 \cdot 7.0000 + y_3^{(1)} &= -7 \Rightarrow \boldsymbol{y_3^{(1)}} = -7.0000
\end{align*}

\[
\boldsymbol{S_y^{(1)}} =
(
\boldsymbol{y_1^{(1)}} = -2.0000,\quad
\boldsymbol{y_2^{(1)}} = 7.0000,\quad
\boldsymbol{y_3^{(1)}} = -7.0000
)
\]

\subsection{Resolvendo $Ux = y$}

\[
Ux =
\begin{bmatrix}
6.0000 & -1.0000 & 12.0000 \\
0      & 2.3333  & -5.0000 \\
0      & 0       & 7.0000
\end{bmatrix}
\begin{bmatrix}
x_1^{(0)} \\
x_2^{(0)} \\
x_3^{(0)}
\end{bmatrix}
=
\begin{bmatrix}
-2.0000 \\
7.0000 \\
-7.0000
\end{bmatrix}
\]

\begin{align*}
x_3^{(1)} &= 0,1428[- 7.0000]\Rightarrow \boldsymbol{x_3^{(1)}} = - 0.9996 \\
x_2^{(1)} &= 0.4285[7.0000 + (5.0000 * 0.7000)] \Rightarrow \boldsymbol{x_2^{(1)}} = 4,4992 \\
x_1^{(1)} &= 0.1666[-2.0000 + 1.0000 - (12.0000 * 0.7000] \Rightarrow \boldsymbol{x_1^{(1)}} = -1.5660
\end{align*}

\[
\boldsymbol{S_x^{(1)}} =
(
\boldsymbol{x_1^{(1)}} = -1.5660,\quad
\boldsymbol{x_2^{(1)}} = 4.4992,\quad
\boldsymbol{x_3^{(1)}} = -1.5660
)
\]

\subsection{Iterações (de k = 0 a k = 5)}
Assim, aplicando esse método para todas as 6 iterações (de k = 0 a k = 5), obteremos: \\

Iteração 1:

\[
\boldsymbol{S_x^{(0)}} =
\left(
\boldsymbol{x_1^{(0)}} = 0.2000, \quad
\boldsymbol{x_2^{(0)}} = 1.0000, \quad
\boldsymbol{x_3^{(0)}} = 0.7000
\right)
\]

Iteração 2:

\[
\boldsymbol{S_x^{(1)}} =
\left(
\boldsymbol{x_1^{(1)}} = -1.5660, \quad
\boldsymbol{x_2^{(1)}} = 4.4992, \quad
\boldsymbol{x_3^{(1)}} = -0.9996
\right)
\]

Iteração 3:

\[
\boldsymbol{S_x^{(2)}} =
\left(
\boldsymbol{x_1^{(2)}} = 2.4135, \quad
\boldsymbol{x_2^{(2)}} = 0.8578, \quad
\boldsymbol{x_3^{(2)}} = -0.9996
\right)
\]

Iteração 4:

\[
\boldsymbol{S_x^{(3)}} =
\left(
\boldsymbol{x_1^{(3)}} = 1.8081, \quad
\boldsymbol{x_2^{(3)}} = 0.8578, \quad
\boldsymbol{x_3^{(3)}} = -0.9996
\right)
\] \\

Iteração 5:

\[
\boldsymbol{S_x^{(4)}} =
\left(
\boldsymbol{x_1^{(4)}} = 1.8081, \quad
\boldsymbol{x_2^{(4)}} = 0.8578, \quad
\boldsymbol{x_3^{(4)}} = -0.9996
\right)
\]

 Iteração 6:

\[
\boldsymbol{S_x^{(5)}} =
\left(
\boldsymbol{x_1^{(5)}} = 1.8081, \quad
\boldsymbol{x_2^{(5)}} = 0.8578, \quad
\boldsymbol{x_3^{(5)}} = -0.9996
\right)
\]

\newpage

\noindent \section{3ª QUESTÃO} (0,5 PONTO): Resolver o sistema de equações utilizando a técnica de Cholesky, admitindo 4 casas decimais nos resultados.

\[
\begin{cases}
5x_1^{(k)} + 1x_2^{(k)} + 7x_3^{(k)} = 2 \\
1x_1^{(k)} + 2x_2^{(k)} + 2x_3^{(k)} = 3 \\
7x_1^{(k)} + 2x_2^{(k)} + 12x_3^{(k)} = 3
\end{cases}
\]
\begin{align*}
S_y^{(0)} &= ( y_1^{(0)} = 0 \quad y_2^{(0)} = 0 \quad y_3^{(0)} = 0 )^{t} \\
S_x^{(0)} &= ( x_1^{(0)} = 0.7 \quad x_2^{(0)} = 0.1 \quad x_3^{(0)} = 0.3 )^{t}
\end{align*}

\textbf{A.1)} Mostrar a metodologia da técnica de Cholesky.

\textbf{A.2)} Fazer 6 soluções.

\vspace{0.5cm}

\subsection{Fundamentação Teórica}

A técnica de solução pelo \textbf{método de Cholesky} serve para resolver sistemas lineares da forma \( Ax = b \), em que a matriz \( A \) é simétrica e definida positiva.

\vspace{0.3cm}
O método consiste em decompor a matriz \( A \) como \( A = L \cdot L^t \), onde \( L \) é uma matriz triangular inferior e \( L^t \) é a transposta de \( L \), ou seja, uma matriz triangular superior. A solução do sistema é feita em duas etapas:

\begin{enumerate}
    \item Resolver \( Ly = b \) por substituição direta.
    \item Resolver \( L^T x = y \) por substituição reversa.
\end{enumerate}

Esse método é mais rápido e preciso do que a eliminação de Gauss para esse tipo de matriz, porque aproveita as características de \( A \) para fazer menos cálculos e reduzir erros numéricos. Isso também significa que ele exige menos esforço do computador, sendo mais eficiente em termos de tempo e uso de recursos.

\subsection{Aplicação do Método de Cholesky}

Para sua aplicação, devem ser atendidas as seguintes condições:

\begin{itemize}
    \item A matriz $A$ dos coeficientes deve ser simétrica: $a_{ij} = a_{ji}$.
    \item A matriz $A$ deve ser positiva definida.
\end{itemize}

\[
A = \begin{bmatrix}
5 & 1 & 7 \\
1 & 2 & 2 \\
7 & 2 & 12
\end{bmatrix}
\quad
A^t = \begin{bmatrix}
5 & 1 & 7 \\
1 & 2 & 2 \\
7 & 2 & 12
\end{bmatrix}
\]

\[
A = A^t \Rightarrow \text{Logo, } A \text{ é simétrica.}
\]

\vspace{0.2cm}
\noindent
\textbf{Obs.:} Neste caso, já iremos assumir diretamente que a matriz é positiva. Porém, caso não tivéssemos certeza dessa informação, poderíamos:

\begin{itemize}
    \item Calcular os autovalores da matriz $A$, que devem ser todos positivos;
    \item Verificar se os \textit{minores principais} são positivos.
\end{itemize}

Caso a matriz não seja positiva definida, e não saibamos disso, a aplicação do método pode falhar — por exemplo, ao tentar calcular a raiz quadrada de um número negativo.

\subsection{Decomposição da matriz $A$ em $G$ e $G^t$}

Satisfazendo as duas condições, podemos prosseguir com a aplicação do método e dividir a matriz $A$ da seguinte forma:

\[
A = G \cdot G^t
\]

Assim, teremos as seguintes matrizes:

\[
A = \begin{bmatrix}
5 & 1 & 7 \\
1 & 2 & 2 \\
7 & 2 & 12
\end{bmatrix},
\quad
G = \begin{bmatrix}
g_{11} & 0 & 0 \\
g_{21} & g_{22} & 0 \\
g_{31} & g_{32} & g_{33}
\end{bmatrix},
\quad
G^t = \begin{bmatrix}
g_{11} & g_{21} & g_{31} \\
0 & g_{22} & g_{32} \\
0 & 0 & g_{33}
\end{bmatrix}
\]

\vspace{0.3cm}
Com base nos elementos de $A$, determinam-se os $g_{ij}$ por aplicação direta das fórmulas de Cholesky, conforme descrito a seguir para cada coluna.

\subsubsection{Coluna 1:}

\[
\begin{bmatrix}
5 \\ 1 \\ 7
\end{bmatrix}
=
\begin{bmatrix}
g_{11} & 0 & 0 \\
g_{21} & g_{22} & 0 \\
g_{31} & g_{32} & g_{33}
\end{bmatrix}
\cdot
\begin{bmatrix}
g_{11} \\ 0 \\ 0
\end{bmatrix}
\]

\[
g_{11}^2 = 5 \Rightarrow \underline {g_{11} = 2,2360}
\]
\[
g_{21} \cdot g_{11} = 1 \Rightarrow \underline {g_{21} = 0,4472}
\]

\[
g_{31} \cdot g_{11} = 7 \Rightarrow \underline {g_{31} = 3,1304}
\]

\subsubsection{Coluna 2}

\[
\begin{bmatrix}
1 \\
2 \\
2
\end{bmatrix}
=
\begin{bmatrix}
g_{11} & 0 & 0 \\
g_{21} & g_{22} & 0 \\
g_{31} & g_{32} & g_{33}
\end{bmatrix}
\cdot
\begin{bmatrix}
g_{21} \\
g_{22} \\
0
\end{bmatrix}
\]

\[
g_{11} \cdot g_{21} = 1 \Rightarrow g_{21} = 0.4472
\]
\[
\text{(Não era necessário calcular } g_{21} \text{ pois já fizemos isso na coluna anterior)}
\]
\[
g_{21}^2 + g_{22}^2 = 2 \Rightarrow \left(\frac{1}{\sqrt{5}}\right)^2 + g_{22}^2 = 2 \Rightarrow g_{22}^2 = 2 - \frac{1}{5} \Rightarrow g_{22} = \frac{\sqrt{9}}{\sqrt{5}} \Rightarrow \underline {g_{22} = 1.3416}
\]
\[
g_{31} \cdot g_{21} + g_{32} \cdot g_{22} = 2 \Rightarrow \frac{7}{\sqrt{5}} \cdot \frac{1}{\sqrt{5}} + g_{32} \cdot \frac{3}{\sqrt{5}} = 2 \Rightarrow \frac{3}{\sqrt{5}} \cdot g_{32} = 2 - \frac{7}{5} \Rightarrow \underline {g_{32} = 0.4472}
\]

\subsubsection{Coluna 3}

\[
\begin{bmatrix}
7 \\
2 \\
12
\end{bmatrix}
=
\begin{bmatrix}
g_{11} & 0 & 0 \\
g_{21} & g_{22} & 0 \\
g_{31} & g_{32} & g_{33}
\end{bmatrix}
\cdot
\begin{bmatrix}
g_{31} \\
g_{32} \\
g_{33}
\end{bmatrix}
\]

\[
g_{11} \cdot g_{31} = 7 \Rightarrow \text{(Já calculamos essas variáveis anteriormente)}
\]
\[
g_{21} \cdot g_{31} + g_{22} \cdot g_{32} = 2 \Rightarrow \text{(Já calculamos essas variáveis anteriormente)}
\]
\[
g_{31}^2 + g_{32}^2 + g_{33}^2 = 12 \Rightarrow \frac{49}{5} + \frac{5}{25} + g_{33}^2 = 12 \Rightarrow \underline {g_{33} = 1.4142}
\]

\subsubsection{Matrizes \( G \) e \( G^T \)}

\[
G =
\begin{bmatrix}
2.2360 & 0 & 0 \\
0.4472 & 1.3416 & 0 \\
3.1304 & 0.4472 & 1.4142
\end{bmatrix}, \quad
G^t =
\begin{bmatrix}
2.2360 & 0.4472 & 3.1304 \\
0 & 1.3416 & 0.4472 \\
0 & 0 & 1.4142
\end{bmatrix}
\]

\vspace{3em}

\noindent No procedimento acima obtivemos todos os valores da matriz \( G \). Com eles, podemos resolver os sistemas para encontrar \( y_j^k \) e \( x_j^k \) conforme as equações abaixo.

\[
\underline{\text{(1)} \quad g_{ij} \cdot y_j^{(k)} = b_i}, \quad i = 1, 2, 3; \quad j = 1, 2, 3
\]

\[
\underline{\text{(2)} \quad g_{ij}^t \cdot x_j^{(k)} = y_i^{(k)}}, \quad i = 1, 2, 3; \quad j = 1, 2, 3
\]

\subsection{Resolução dos sistemas para \(y_j^{(k)}\) e \(x_j^{(k)}\)}

\subsubsection{Coeficientes \(y_j^{(k)}\)}

\noindent
O sistema de equações é dado por:

\[
g_{ij} \cdot y_j^{(k)} = b_i, \quad i = 1,2,3;\quad j = 1,2,3
\]

\noindent
Na forma de multiplicação matricial chegamos ao sistema abaixo:

\[
\begin{bmatrix}
g_{11} & 0      & 0 \\
g_{21} & g_{22} & 0 \\
g_{31} & g_{32} & g_{33}
\end{bmatrix}
\cdot
\begin{bmatrix}
y_1^{(k)} \\
y_2^{(k)} \\
y_3^{(k)}
\end{bmatrix}
=
\begin{bmatrix}
b_1 \\
b_2 \\
b_3
\end{bmatrix}
\]

\noindent
Resolvendo a multiplicação matricial, obtemos o seguinte sistema de equações:
\noindent
\begin{align*}
&\text{(1)} \quad g_{11} \cdot y_1^{(k)} = b_1 \\
&\text{(2)} \quad g_{21} \cdot y_1^{(k)} + g_{22} \cdot y_2^{(k)} = b_2 \\
&\text{(3)} \quad g_{31} \cdot y_1^{(k)} + g_{32} \cdot y_2^{(k)} + g_{33} \cdot y_3^{(k)} = b_3
\end{align*}

\noindent
\textbf{Agora, isolando as variáveis \( y_1^{(k)}, y_2^{(k)} \) e \( y_3^{(k)} \):}

\[
\left\{
\begin{aligned}
y_1^{(k)} &= \frac{b_1}{g_{11}} \\
y_2^{(k)} &= \frac{1}{g_{22}} \left[ b_2 - \left(g_{21} \cdot y_1^{(k)}\right) \right] \\
y_3^{(k)} &= \frac{1}{g_{33}} \left[ b_3 - \left(g_{31} \cdot y_1^{(k)} + g_{32} \cdot y_2^{(k)}\right) \right]
\end{aligned}
\right.
\]
\[\]
\[\]
\noindent
Como a questão nos dá uma solução inicial:

\[
S_y^{(0)} = \left( y_1^{(0)} = 0 \quad y_2^{(0)} = 0 \quad y_3^{(0)} = 0 \right)^{t}
\]

\noindent
Podemos substituí-los juntamente com os valores da matriz \( G \) para obter \( y_j^{(k + 1)} \).

\[
\left\{
\begin{aligned}
y_1^{(k + 1)} &= \frac{2}{2.2360} = 0.8944 \\
y_2^{(k + 1)} &= \frac{1}{1.3416} \left[ 3 - \left(0.4472 \cdot 0\right) \right] = 2.2360 \\
y_3^{(k + 1)} &= \frac{1}{1.4142} \left[ 3 - \left(3.1304 \cdot 0 + 0.4472 \cdot 0\right) \right] = 2.1213
\end{aligned}
\right.
\]

Logo, temos a solução:

\[
\boldsymbol{S_y^{(k + 1)}} = \left( \boldsymbol{y_1^{(k + 1)}} = \mathbf{0.8944} \quad \boldsymbol{y_2^{(k + 1)}} = \mathbf{2.2360} \quad \boldsymbol{y_3^{(k + 1)}} = \mathbf{2.1213} \right)^{t}
\]

Com isso, podemos resolver a matriz triangular superior para encontrar valores de \( x_j(k) \).

\vspace{0.3cm}

\uline{Cabe destacar que o conjunto solução \( S_y^{(k + 1)} = [\ldots] \), obtido acima, permanece constante. Ou seja, será o mesmo utilizado em todas as iterações subsequentes.}

\subsubsection{Coeficientes \(x_j^{(k)}\)}

Montando o sistema.
\[
g_{ij} \cdot y_j^{(k)} = b_i, \quad i = 1, 2, 3; \quad j = 1, 2, 3
\]

Sistema obtido pelo desenvolvimento das matrizes:
\[
\begin{bmatrix}
g_{11} & g_{21} & g_{31} \\
0 & g_{22} & g_{32} \\
0 & 0 & g_{33}
\end{bmatrix}
\cdot
\begin{bmatrix}
x_1^{(k)} \\
x_2^{(k)} \\
x_3^{(k)}
\end{bmatrix}
=
\begin{bmatrix}
y_1^{(k)} \\
y_2^{(k)} \\
y_3^{(k)}
\end{bmatrix}
\]

Fazendo os produtos, chegamos às seguintes equações:
\begin{align*}
&\text{(1)} \quad g_{11} \cdot x_1^{(k)} + g_{21} \cdot x_2^{(k)} + g_{31} \cdot x_3^{(k)} = y_1^{(k)} \\
&\text{(2)} \quad g_{22} \cdot x_2^{(k)} + g_{32} \cdot x_3^{(k)} = y_2^{(k)} \\
&\text{(3)} \quad g_{33} \cdot x_3^{(k)} = y_3^{(k)}
\end{align*}
\[\]
\textbf{Isolando $x_j^{(k)}$ em cada equação:}

\vspace{0.5cm}

\[
\left\{
\begin{aligned}
x_1^{(k+1)} &= \frac{1}{g_{11}} \left[ y_1^{(k+1)} - \left(g_{21} \cdot x_2^{(k)} + g_{31} \cdot x_3^{(k)}\right) \right] \\
x_2^{(k+1)} &= \frac{1}{g_{22}} \left[ y_2^{(k+1)} - \left(g_{32} \cdot x_3^{(k)}\right) \right] \\
x_3^{(k+1)} &= \frac{1}{g_{33}} \cdot y_3^{(k+1)} \\
\end{aligned}
\right.
\]

\vspace{0.5cm}

Com o sistema acima basta introduzir os valores de \textbf{$k$} (para cada iteração) e substituir os valores numéricos da matriz $G$ e do conjunto solução $S_y^{(k + 1)} = [\ldots]$ já calculados anteriormente.

Dessa forma encontramos um novo conjunto solução para $x_j^{k+1}$.

Realizaremos o processo 5 vezes ($k=0,1,2,3,4$) para obtermos 5 soluções $x_j^{(k)}$.

O enunciado pede 6 soluções, mas já entregou 1, por isso faremos apenas mais 5 iterações.

\subsection{Iterações e Soluções}

\subsubsection{1ª iteração - k = 0}

Substituindo os valores de $k$ e $g_{ij}$:

\[
\left\{
\begin{aligned}
x_1^{(0+1)} &= \frac{1}{2.2360} \left[ y_1^{(0+1)} - \left(0.4472 \cdot x_2^{(0)} + 3.1304 \cdot x_3^{(0)}\right) \right] \\
x_2^{(0+1)} &= \frac{1}{1.3416} \left[ y_2^{(0+1)} - \left(0.4472 \cdot x_3^{(0)}\right) \right] \\
x_3^{(0+1)} &= \frac{1}{1.4142} \cdot y_3^{(0+1)} \\
\end{aligned}
\right.
\]

Substituindo os valores de $x_j^{0}$ e $y_j$:

\[
\left\{
\begin{aligned}
x_1^{(1)} &= \frac{1}{2.2360} \left[ 0.8944 - \left(0.4472 \cdot 0.1000 + 3.1304 \cdot 0.3000\right) \right] \\
x_2^{(1)} &= \frac{1}{1.3416} \left[ 2.2360 - \left(0.4472 \cdot 0.3000\right) \right] \\
x_3^{(1)} &= \frac{1}{1.4142} \cdot 2.1213 \\
\end{aligned}
\right.
\]

Realizando as operações:
\[
\left\{
\begin{aligned}
x_1^{(1)} &= - 0.0400 \\
x_2^{(1)} &= 1.5667 \\
x_3^{(1)} &= 1.5000 \\
\end{aligned}
\right.
\]

Conjunto solução:
\begin{align*}
S_x^{(1)} &= ( x_1^{(1)} = -0.0400 \quad x_2^{(1)} = 1.5667 \quad x_3^{(1)} = 1.5000 )^{t}
\end{align*}

\subsubsection{2ª iteração - k = 1}

Substituindo os valores de $k$ e $g_{ij}$:

\[
\left\{
\begin{aligned}
x_1^{(1+1)} &= \frac{1}{2.2360} \left[ y_1^{(1+1)} - \left(0.4472 \cdot x_2^{(1)} + 3.1304 \cdot x_3^{(1)}\right) \right] \\
x_2^{(1+1)} &= \frac{1}{1.3416} \left[ y_2^{(1+1)} - \left(0.4472 \cdot x_3^{(1)}\right) \right] \\
x_3^{(1+1)} &= \frac{1}{1.4142} \cdot y_3^{(1+1)} \\
\end{aligned}
\right.
\]

Substituindo os valores de $x_j^{1}$ e $y_j$:

\[
\left\{
\begin{aligned}
x_1^{(2)} &= \frac{1}{2.2360} \left[ 0.8944 - \left(0.4472 \cdot 1.5667 + 3.1304 \cdot 1.5000\right) \right] \\
x_2^{(2)} &= \frac{1}{1.3416} \left[ 2.2360 - \left(0.4472 \cdot 1.5000\right) \right] \\
x_3^{(2)} &= \frac{1}{1.4142} \cdot 2.1213 \\
\end{aligned}
\right.
\]

Realizando as operações:
\[
\left\{
\begin{aligned}
x_1^{(1)} &= - 2.0133 \\
x_2^{(1)} &= 1.1667 \\
x_3^{(1)} &= 1.5000 \\
\end{aligned}
\right.
\]

Conjunto solução:
\begin{align*}
S_x^{(2)} &= ( x_1^{(2)} = -2.0133 \quad x_2^{(2)} = 1.1667 \quad x_3^{(2)} = 1.5000 )^{t}
\end{align*}

\subsubsection{3ª iteração - k = 2}

Substituindo os valores de $k$ e $g_{ij}$:

\[
\left\{
\begin{aligned}
x_1^{(2+1)} &= \frac{1}{2.2360} \left[ y_1^{(2+1)} - \left(0.4472 \cdot x_2^{(2)} + 3.1304 \cdot x_3^{(2)}\right) \right] \\
x_2^{(2+1)} &= \frac{1}{1.3416} \left[ y_2^{(2+1)} - \left(0.4472 \cdot x_3^{(2)}\right) \right] \\
x_3^{(2+1)} &= \frac{1}{1.4142} \cdot y_3^{(2+1)} \\
\end{aligned}
\right.
\]

Substituindo os valores de $x_j^{2}$ e $y_j$:

\[
\left\{
\begin{aligned}
x_1^{(3)} &= \frac{1}{2.2360} \left[ 0.8944 - \left(0.4472 \cdot 1.1667 + 3.1304 \cdot 1.5000\right) \right] \\
x_2^{(3)} &= \frac{1}{1.3416} \left[ 2.2360 - \left(0.4472 \cdot 1.5000\right) \right] \\
x_3^{(3)} &= \frac{1}{1.4142} \cdot 2.1213 \\
\end{aligned}
\right.
\]

Realizando as operações:
\[
\left\{
\begin{aligned}
x_1^{(3)} &= - 1.9334 \\
x_2^{(3)} &= 1.1667 \\
x_3^{(3)} &= 1.5000 \\
\end{aligned}
\right.
\]

Conjunto solução:
\begin{align*}
S_x^{(3)} &= ( x_1^{(3)} = - 1.9334 \quad x_2^{(3)} = 1.1667 \quad x_3^{(3)} = 1.5000 )^{t}
\end{align*}

\subsubsection{4ª iteração - k = 3}

Substituindo os valores de $k$ e $g_{ij}$:

\[
\left\{
\begin{aligned}
x_1^{(3+1)} &= \frac{1}{2.2360} \left[ y_1^{(3+1)} - \left(0.4472 \cdot x_2^{(3)} + 3.1304 \cdot x_3^{(3)}\right) \right] \\
x_2^{(3+1)} &= \frac{1}{1.3416} \left[ y_2^{(3+1)} - \left(0.4472 \cdot x_3^{(3)}\right) \right] \\
x_3^{(3+1)} &= \frac{1}{1.4142} \cdot y_3^{(3+1)} \\
\end{aligned}
\right.
\]

Substituindo os valores de $x_j^{3}$ e $y_j$:

\[
\left\{
\begin{aligned}
x_1^{(4)} &= \frac{1}{2.2360} \left[ 0.8944 - \left(0.4472 \cdot 1.1667 + 3.1304 \cdot 1.5000\right) \right] \\
x_2^{(4)} &= \frac{1}{1.3416} \left[ 2.2360 - \left(0.4472 \cdot 1.5000\right) \right] \\
x_3^{(4)} &= \frac{1}{1.4142} \cdot 2.1213 \\
\end{aligned}
\right.
\]

Realizando as operações:
\[
\left\{
\begin{aligned}
x_1^{(4)} &= - 1.9334 \\
x_2^{(4)} &= 1.1667 \\
x_3^{(4)} &= 1.5000 \\
\end{aligned}
\right.
\]

Conjunto solução:
\begin{align*}
S_x^{(4)} &= ( x_1^{(4)} = - 1.9334 \quad x_2^{(4)} = 1.1667 \quad x_3^{(4)} = 1.5000 )^{t}
\end{align*}

\subsubsection{5ª iteração - k = 4}

Substituindo os valores de $k$ e $g_{ij}$:

\[
\left\{
\begin{aligned}
x_1^{(4+1)} &= \frac{1}{2.2360} \left[ y_1^{(4+1)} - \left(0.4472 \cdot x_2^{(4)} + 3.1304 \cdot x_3^{(4)}\right) \right] \\
x_2^{(4+1)} &= \frac{1}{1.3416} \left[ y_2^{(4+1)} - \left(0.4472 \cdot x_3^{(4)}\right) \right] \\
x_3^{(4+1)} &= \frac{1}{1.4142} \cdot y_3^{(4+1)} \\
\end{aligned}
\right.
\]

Substituindo os valores de $x_j^{4}$ e $y_j$:

\[
\left\{
\begin{aligned}
x_1^{(5)} &= \frac{1}{2.2360} \left[ 0.8944 - \left(0.4472 \cdot 1.1667 + 3.1304 \cdot 1.5000\right) \right] \\
x_2^{(5)} &= \frac{1}{1.3416} \left[ 2.2360 - \left(0.4472 \cdot 1.5000\right) \right] \\
x_3^{(5)} &= \frac{1}{1.4142} \cdot 2.1213 \\
\end{aligned}
\right.
\]

Realizando as operações:
\[
\left\{
\begin{aligned}
x_1^{(5)} &= - 1.9334 \\
x_2^{(5)} &= 1.1667 \\
x_3^{(5)} &= 1.5000 \\
\end{aligned}
\right.
\]

Conjunto solução:
\begin{align*}
S_x^{(5)} &= ( x_1^{(5)} = - 1.9334 \quad x_2^{(5)} = 1.1667 \quad x_3^{(5)} = 1.5000 )^{t}
\end{align*}

\subsection{Soluções}
\begin{align*}
S_x^{(0)} &= ( x_1^{(0)} = 0.7 \quad x_2^{(0)} = 0.1 \quad x_3^{(0)} = 0.3 )^{t} \\
S_x^{(1)} &= ( x_1^{(1)} = -0.0400 \quad x_2^{(1)} = 1.5667 \quad x_3^{(1)} = 1.5000 )^{t} \\
S_x^{(2)} &= ( x_1^{(2)} = -2.0133 \quad x_2^{(2)} = 1.1667 \quad x_3^{(2)} = 1.5000 )^{t} \\
S_x^{(3)} &= ( x_1^{(3)} = - 1.9334 \quad x_2^{(3)} = 1.1667 \quad x_3^{(3)} = 1.5000 )^{t} \\
S_x^{(4)} &= ( x_1^{(4)} = - 1.9334 \quad x_2^{(4)} = 1.1667 \quad x_3^{(4)} = 1.5000 )^{t} \\
S_x^{(5)} &= ( x_1^{(5)} = - 1.9334 \quad x_2^{(5)} = 1.1667 \quad x_3^{(5)} = 1.5000 )^{t}
\end{align*}

\end{document}
